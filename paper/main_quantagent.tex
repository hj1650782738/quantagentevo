%%%%%%%% ICML 2026 EXAMPLE LATEX SUBMISSION FILE %%%%%%%%%%%%%%%%%

\documentclass{article}

% Recommended, but optional, packages for figures and better typesetting:
\usepackage{microtype}
\usepackage{graphicx}
\usepackage{subcaption}


% hyperref makes hyperlinks in the resulting PDF.
% If your build breaks (sometimes temporarily if a hyperlink spans a page)
% please comment out the following usepackage line and replace
% \usepackage{icml2026} with \usepackage[nohyperref]{icml2026} above.
\usepackage{hyperref}


% Attempt to make hyperref and algorithmic work together better:
\newcommand{\theHalgorithm}{\arabic{algorithm}}

% Use the following line for the initial blind version submitted for review:
\usepackage{icml2026}

% For preprint, use
% \usepackage[preprint]{icml2026}

% If accepted, instead use the following line for the camera-ready submission:
% \usepackage[accepted]{icml2026}

\usepackage{amsmath}
\usepackage{amssymb}
\usepackage{mathtools}
\usepackage{amsthm}


%额外加载包
\usepackage{/latex_package/algorithm}
\usepackage{/latex_package/algorithmic}
% \usepackage{/latex_package/fancyhdr}

% if you use cleveref..
\usepackage[capitalize,noabbrev]{cleveref}
\usepackage{threeparttable} % threeparttable + tablenotes
\usepackage[table]{xcolor}  % \rowcolor[RGB]{...}
\usepackage{multirow} 
\usepackage{array} 
\usepackage{makecell}
\usepackage{booktabs} % for professional tables
\usepackage{xspace}
%附录B用的
\usepackage{xcolor}
\usepackage{tcolorbox}
\tcbuselibrary{skins,breakable}
\usepackage{tikz}
\usepackage{fontawesome5}
\usepackage{booktabs}
% \usepackage{titletoc}
% \usepackage{etoolbox}

% \pretocmd{\section}
%   {\addcontentsline{appendices}{section}{\thesection\quad #1}}
%   {}{}

\definecolor{headerblue}{RGB}{41, 128, 185}
\definecolor{metricgreen}{RGB}{39, 174, 96}
\definecolor{alertred}{RGB}{192, 57, 43}
\definecolor{warningyellow}{RGB}{243, 156, 18}
\definecolor{lightgray}{RGB}{245, 245, 245}
\definecolor{darkgray}{RGB}{52, 73, 94}
\definecolor{mutationpurple}{RGB}{142, 68, 173}
\definecolor{crossoverblue}{RGB}{52, 152, 219}



%%%%%%%%%%%%%%%%%%%%%%%%%%%%%%%%
% THEOREMS
%%%%%%%%%%%%%%%%%%%%%%%%%%%%%%%%
\theoremstyle{plain}
\newtheorem{theorem}{Theorem}[section]
\newtheorem{proposition}[theorem]{Proposition}
\newtheorem{lemma}[theorem]{Lemma}
\newtheorem{corollary}[theorem]{Corollary}
\theoremstyle{definition}
\newtheorem{definition}[theorem]{Definition}
\newtheorem{assumption}[theorem]{Assumption}
\theoremstyle{remark}
\newtheorem{remark}[theorem]{Remark}

% Todonotes is useful during development; simply uncomment the next line
%    and comment out the line below the next line to turn off comments
%\usepackage[disable,textsize=tiny]{todonotes}
\usepackage[textsize=tiny]{todonotes}

% The \icmltitle you define below is probably too long as a header.
% Therefore, a short form for the running title is supplied here:
\icmltitlerunning{Submission and Formatting Instructions for ICML 2026}

\newcommand{\method}{QuantaAlpha\xspace}

\begin{document}

\twocolumn[
  \icmltitle{QuantaAlpha: An Evolutionary Framework for LLM-Driven\\ Interpretable Alpha Mining}

  % It is OKAY to include author information, even for blind submissions: the
  % style file will automatically remove it for you unless you've provided
  % the [accepted] option to the icml2026 package.

  % List of affiliations: The first argument should be a (short) identifier you
  % will use later to specify author affiliations Academic affiliations
  % should list Department, University, City, Region, Country Industry
  % affiliations should list Company, City, Region, Country

  % You can specify symbols, otherwise they are numbered in order. Ideally, you
  % should not use this facility. Affiliations will be numbered in order of
  % appearance and this is the preferred way.
  \icmlsetsymbol{equal}{*}

  \begin{icmlauthorlist}
    \icmlauthor{Firstname1 Lastname1}{equal,yyy}
    \icmlauthor{Firstname2 Lastname2}{equal,yyy,comp}
    \icmlauthor{Firstname3 Lastname3}{comp}
    \icmlauthor{Firstname4 Lastname4}{sch}
    \icmlauthor{Firstname5 Lastname5}{yyy}
    \icmlauthor{Firstname6 Lastname6}{sch,yyy,comp}
    \icmlauthor{Firstname7 Lastname7}{comp}
    %\icmlauthor{}{sch}
    \icmlauthor{Firstname8 Lastname8}{sch}
    \icmlauthor{Firstname8 Lastname8}{yyy,comp}
    %\icmlauthor{}{sch}
    %\icmlauthor{}{sch}
  \end{icmlauthorlist}

  \icmlaffiliation{yyy}{Department of XXX, University of YYY, Location, Country}
  \icmlaffiliation{comp}{Company Name, Location, Country}
  \icmlaffiliation{sch}{School of ZZZ, Institute of WWW, Location, Country}

  \icmlcorrespondingauthor{Firstname1 Lastname1}{first1.last1@xxx.edu}
  \icmlcorrespondingauthor{Firstname2 Lastname2}{first2.last2@www.uk}

  % You may provide any keywords that you find helpful for describing your
  % paper; these are used to populate the "keywords" metadata in the PDF but
  % will not be shown in the document
  \icmlkeywords{Machine Learning, ICML}

  \vskip 0.3in
]

% this must go after the closing bracket ] following \twocolumn[ ...

% This command actually creates the footnote in the first column listing the
% affiliations and the copyright notice. The command takes one argument, which
% is text to display at the start of the footnote. The \icmlEqualContribution
% command is standard text for equal contribution. Remove it (just {}) if you
% do not need this facility.

% Use ONE of the following lines. DO NOT remove the command.
% If you have no special notice, KEEP empty braces:
\printAffiliationsAndNotice{}  % no special notice (required even if empty)
% Or, if applicable, use the standard equal contribution text:
% \printAffiliationsAndNotice{\icmlEqualContribution}

\begin{abstract}
Factor mining in quantitative investment is notoriously difficult due to low signal-to-noise ratios, strong non-stationarity, and frequent market regime shifts, where single-period in-sample backtesting is insufficient to establish robustness. Recent LLM- and agent-based approaches improve automation but struggle to conduct constrained, multi-round searches that prioritize out-of-sample stability under realistic financial constraints.
We propose \textbf{QuantaAlpha}, a hypothesis-centric self-evolving framework for factor mining. The framework formulates factor discovery as a constrained black-box search problem and treats research hypotheses as evolutionary individuals. A planning module initializes hypothesis populations, which are iteratively refined through orthogonal mutation, crossover, and hybrid selection under explicit complexity constraints in a nested closed-loop process.
Experiments on the CSI 300 stock universe demonstrate that QuantaAlpha consistently yields more stable and generalizable factor pools than existing baselines. Additional analyses on ablation, alpha decay, and factor diversity further validate the effectiveness and robustness of the proposed framework.
\end{abstract}



%main paper
\input{main_section/1.introduction}
\input{main_section/2.related_work}
\input{main_section/3.problem_formulation}
\input{main_section/4.method}
\input{main_section/5.experiments}
% \input{main_section/6.further_analysis}
\input{main_section/6.conclusion}
\input{main_section/7.acknowledgement}

%%%%%%%%%%%%%%%%%%%%%%%%
%%%%%%%%%%NOTE%%%%%%%%%%
%%%%%%%%%%%%%%%%%%%%%%%%
%插入表和图分别放 tables 和 images 文件夹里



%reference
\bibliography{reference}
\bibliographystyle{icml2026}



%%%%%%%%%%%%%%%%%%%%%%%%%%%%%%%%%%%%%%%%%%%%%%%%%%%%%%%%%%%%%%%%%%%%%%%%%%%%%%%
%%%%%%%%%%%%%%%%%%%%%%%%%%%%%%%%%%%%%%%%%%%%%%%%%%%%%%%%%%%%%%%%%%%%%%%%%%%%%%%
% APPENDIX
%%%%%%%%%%%%%%%%%%%%%%%%%%%%%%%%%%%%%%%%%%%%%%%%%%%%%%%%%%%%%%%%%%%%%%%%%%%%%%%
%%%%%%%%%%%%%%%%%%%%%%%%%%%%%%%%%%%%%%%%%%%%%%%%%%%%%%%%%%%%%%%%%%%%%%%%%%%%%%%
\newpage
\appendix
\onecolumn

%附录自定义章节,名称
% Appendix A: Experiment Settings

\section{Experiment Settings}
\label{sec:appendix_experiment_details}

This section provides details of our experimental setup, including computational infrastructure, evaluation metrics, backtesting setup, and baselines.

%%%%%%%%%%%%%%%%%%%%%%%%%%%%%%%%%%%%%%%%%%%%%%%%%%%%%%%%%%%%%%%%%%%%%%%%%%%%%%
\subsection{Computational Infrastructure}
%%%%%%%%%%%%%%%%%%%%%%%%%%%%%%%%%%%%%%%%%%%%%%%%%%%%%%%%%%%%%%%%%%%%%%%%%%%%%%

All experiments are conducted on a high-performance computing server with the configuration shown in Table~\ref{tab:hardware_config}.

\begin{table}[!htbp]
\centering
\caption{Hardware Configuration for Experimental Infrastructure}
\label{tab:hardware_config}
\begin{tabular}{ll}
\toprule
\textbf{Component} & \textbf{Specification} \\
\midrule
CPU & 2 $\times$ Intel Xeon Gold 6348 (56 cores / 112 threads) \\
Memory & 755 GB DDR4 RAM \\
GPU & 4 $\times$ NVIDIA RTX 6000 Ada (48 GB VRAM each) \\
\bottomrule
\end{tabular}
\end{table}

%%%%%%%%%%%%%%%%%%%%%%%%%%%%%%%%%%%%%%%%%%%%%%%%%%%%%%%%%%%%%%%%%%%%%%%%%%%%%%
\subsection{Evaluation Metrics}
\label{appendix:A.2}
%%%%%%%%%%%%%%%%%%%%%%%%%%%%%%%%%%%%%%%%%%%%%%%%%%%%%%%%%%%%%%%%%%%%%%%%%%%%%%

We evaluate predictive performance using two categories of metrics: factor predictive power and strategy-level performance.

Without loss of generality, the bar notation $\bar{\cdot}$ denotes the mean, and $\sigma(\cdot)$ denotes the standard deviation.

\begin{itemize}
    \setlength{\itemsep}{-2pt}
    \setlength{\parsep}{0pt}
    \setlength{\topsep}{0pt}
    \setlength{\partopsep}{0pt} 
    \item \textbf{Information Coefficient (IC)}: Pearson correlation between factor values $\mathbf{f}_t$ and future returns $\mathbf{r}_{t+1}$:
    \begin{equation*}
        \operatorname{IC}_t = \frac{(\mathbf{f}_t - \bar{f}_t \mathbf{1})^\top (\mathbf{r}_{t+1} - \bar{r}_{t+1} \mathbf{1})}{\|\mathbf{f}_t - \bar{f}_t \mathbf{1}\|_2 \cdot \|\mathbf{r}_{t+1} - \bar{r}_{t+1} \mathbf{1}\|_2},
    \end{equation*}
    where $\mathbf{1}$ denotes a column vector of ones.
    
    \item \textbf{ICIR}: Information ratio of IC, measuring consistency: $\operatorname{ICIR} = \overline{\operatorname{IC}} / \sigma(\operatorname{IC})$.
    
    \item \textbf{Rank IC}: Spearman correlation using rank vectors $\tilde{\mathbf{f}}_t = \operatorname{rank}(\mathbf{f}_t)$ and $\tilde{\mathbf{r}}_{t+1} = \operatorname{rank}(\mathbf{r}_{t+1})$:
    \begin{equation*}
        \operatorname{RankIC}_t = \frac{(\tilde{\mathbf{f}}_t - \bar{\tilde{f}}_t \mathbf{1})^\top (\tilde{\mathbf{r}}_{t+1} - \bar{\tilde{r}}_{t+1} \mathbf{1})}{\|\tilde{\mathbf{f}}_t - \bar{\tilde{f}}_t \mathbf{1}\|_2 \cdot \|\tilde{\mathbf{r}}_{t+1} - \bar{\tilde{r}}_{t+1} \mathbf{1}\|_2},
    \end{equation*}
    where $\operatorname{rank}(\cdot)$ is the rank function applied element-wise to its input vector in ascending order.
    
    \item \textbf{Rank ICIR}: Information ratio of Rank IC: $\operatorname{RankICIR} = \overline{\operatorname{RankIC}} / \sigma(\operatorname{RankIC})$.
\end{itemize}

All strategy metrics are computed on \textit{excess returns after transaction costs}, where $r_{\text{excess},t} = r_{\text{portfolio},t} - r_{\text{benchmark},t} - c_{\text{transaction},t}$. Here, $r_{\text{portfolio},t}$ represents the return of the strategy portfolio, $r_{\text{benchmark},t}$ denotes the return of the market benchmark, and $c_{\text{transaction},t}$ accounts for the transaction costs incurred at time $t$.

\begin{itemize}
    \setlength{\itemsep}{-2pt}
    \setlength{\parsep}{0pt}
    \setlength{\topsep}{0pt}
    \setlength{\partopsep}{0pt}
    \item \textbf{Information Ratio ($\operatorname{IR}$)}: $\operatorname{IR} = (\overline{r_{\text{excess}}} / \sigma(r_{\text{excess}})) \times \sqrt{252}$.
    \item \textbf{Annualized Return ($\operatorname{ARR}$)}: Annualized excess return over benchmark.
    \item \textbf{Maximum Drawdown ($\operatorname{MDD}$)}: Largest peak-to-trough decline in cumulative excess returns.
    \item \textbf{Calmar Ratio ($\operatorname{CR}$)}: $\operatorname{CR} = \operatorname{ARR} / |\operatorname{MDD}|$.
\end{itemize}

%%%%%%%%%%%%%%%%%%%%%%%%%%%%%%%%%%%%%%%%%%%%%%%%%%%%%%%%%%%%%%%%%%%%%%%%%%%%%%
\subsection{Backtesting Setup}
\label{subsec:backtest_setup}
%%%%%%%%%%%%%%%%%%%%%%%%%%%%%%%%%%%%%%%%%%%%%%%%%%%%%%%%%%%%%%%%%%%%%%%%%%%%%%

Backtesting is conducted using the Qlib framework across the CSI 300, CSI 500, and S\&P 500 indices, with the data split detailed in Table~\ref{tab:data_split}. Factor construction utilizes six basic features—\texttt{open}, \texttt{high}, \texttt{low}, \texttt{close}, \texttt{volume}, and \texttt{vwap}—to predict the next-day return, defined as $y_t = P_{t+2}^{\text{close}} / P_{t+1}^{\text{close}} - 1$, where $P_{t}^{\text{close}}$ denotes the closing price at time $t$. To ensure robustness against outliers, the preprocessing pipeline includes forward-filling missing values, replacing infinite values, dropping samples with missing labels, and applying cross-sectional rank normalization (CSRankNorm) to both features and labels.

\begin{table}[!htbp]
\centering
\caption{Data Split Periods for Train, Validation, and Test Sets across All Markets}
\label{tab:data_split}
\begin{tabular}{lccc}
\toprule
\textbf{Market} & \textbf{Train} & \textbf{Valid} & \textbf{Test} \\
\midrule
CSI 300 & 2016-01-01--2020-12-31 & 2021-01-01--2021-12-31 & 2022-01-01--2025-12-26 \\
CSI 500 & 2016-01-01--2020-12-31 & 2021-01-01--2021-12-31 & 2022-01-01--2025-12-26 \\
S\&P 500 & 2016-01-01--2020-12-31 & 2021-01-01--2021-12-31 & 2022-01-01--2025-12-26 \\
\bottomrule
\end{tabular}
\end{table}

%%%%%%%%%%%%%%%%%%%%%%%%%%%%%%%%%%%%%%%%%%%%%%%%%%%%%%%%%%%%%%%%%%%%%%%%%%%%%%
\subsection{Baselines}
\label{appendix:A.5}
%%%%%%%%%%%%%%%%%%%%%%%%%%%%%%%%%%%%%%%%%%%%%%%%%%%%%%%%%%%%%%%%%%%%%%%%%%%%%%

We benchmark against four categories: (1) \textbf{ML models}: Linear Regression (Linear), Multi-Layer Perceptron (MLP), and gradient boosting decision trees including LightGBM, XGBoost, and CatBoost, along with DoubleEnsemble, an ensemble method for financial time series; (2) \textbf{Deep learning}: Recurrent networks such as Gated Recurrent Unit (GRU) and Long Short-Term Memory (LSTM), the attention-based Transformer, and Temporal Routing Adaptor (TRA); (3) \textbf{Classical factors}: Alpha158 and Alpha360, which are widely used sets of technical factors derived from price and volume; (4) \textbf{LLM agents}: RD-Agent and AlphaAgent, which utilize large language models for automated factor mining.

%%%%%%%%%%%%%%%%%%%%%%%%%%%%%%%%%%%%%%%%%%%%%%%%%%%%%%%%%%%%%%%%%%%%%%%%%%%%%%
%%%%%%%%%%%%%%%%%%%%%%%%%%%%%%%%%%%%%%%%%%%%%%%%%%%%%%%%%%%%%%%%%%%%%%%%%%%%%%

\section{Algorithm Configuration}
\label{sec:appendix_algorithm}

This section details the evolution algorithm parameters, factor constraints, and trading strategy configuration.

%%%%%%%%%%%%%%%%%%%%%%%%%%%%%%%%%%%%%%%%%%%%%%%%%%%%%%%%%%%%%%%%%%%%%%%%%%%%%%
\subsection{Evolution Algorithm}
\label{subsec:evolution_config}
%%%%%%%%%%%%%%%%%%%%%%%%%%%%%%%%%%%%%%%%%%%%%%%%%%%%%%%%%%%%%%%%%%%%%%%%%%%%%%

QuantaAlpha employs an evolutionary algorithm with mutation and crossover operations. Key hyperparameters are shown in Table~\ref{tab:evolution_params}. The process alternates between mutation (exploring orthogonal strategies) and crossover (combining successful trajectories) phases across multiple rounds. LightGBM is used as the downstream model for factor-based prediction.

\begin{table}[!htbp]
\centering
\caption{Hyperparameters for the QuantaAlpha Evolution Algorithm}
\label{tab:evolution_params}
\begin{tabular}{llp{8.5cm}}
\toprule
\textbf{Parameter} & \textbf{Value} & \textbf{Description} \\
\midrule
\multicolumn{2}{c}{\textit{Planning Phase}} \\
\midrule
num\_directions & 10 & Parallel exploration directions \\
max\_planning\_attempts & 5 & Max LLM retry attempts \\
\midrule
\multicolumn{2}{c}{\textit{Evolution Phase}} \\
\midrule
max\_rounds & 11 & Max evolution rounds \\
crossover\_size & 2 & Parents per crossover \\
crossover\_n & 10 & Offspring per round \\
parent\_selection & best & Selection strategy \\
\midrule
\multicolumn{2}{c}{\textit{Execution Phase}} \\
\midrule
max\_loops & 11 & Max iterations per trajectory \\
steps\_per\_loop & 5 & Steps: propose$\to$construct$\to$calculate$\to$backtest$\to$feedback \\
factors\_per\_hypothesis & 3 & Factor expressions per hypothesis \\
\bottomrule
\end{tabular}
\end{table}

%%%%%%%%%%%%%%%%%%%%%%%%%%%%%%%%%%%%%%%%%%%%%%%%%%%%%%%%%%%%%%%%%%%%%%%%%%%%%%
\label{subsec:factor_complexity}
%%%%%%%%%%%%%%%%%%%%%%%%%%%%%%%%%%%%%%%%%%%%%%%%%%%%%%%%%%%%%%%%%%%%%%%%%%%%%%

To prevent overfitting and ensure interpretability, factor expressions—built using the operators listed in Table~\ref{tab:factor_operators}—are restricted by the following constraints: symbol length $\leq$ 250 characters, base features $\leq$ 6, free arguments ratio $<$ 50\%, and duplicate subtree size $\leq$ 5. Details of the associated complexity penalty formulation are discussed in the main text.

\begin{table}[!htbp]
\centering
\caption{List of Supported Operators for Factor Construction}
\label{tab:factor_operators}
\small
\setlength{\tabcolsep}{4pt}
\begin{tabular}{p{2.3cm} p{8cm} @{\hspace{0.6cm}} p{5.5cm}}
\toprule
\textbf{Category} & \textbf{Operators} & \textbf{Description} \\
\midrule
Time-Series & \texttt{DELTA}, \texttt{DELAY}, \texttt{TS\_MEAN}, \texttt{TS\_STD}, \texttt{TS\_VAR}, \texttt{TS\_MAX}, \texttt{TS\_MIN}, \texttt{TS\_SUM}, \texttt{TS\_RANK}, \texttt{TS\_CORR}, \texttt{TS\_COVARIANCE}, \texttt{TS\_ARGMAX}, \texttt{TS\_ARGMIN}, \texttt{TS\_SKEW}, \texttt{TS\_KURT}, \texttt{TS\_PCTCHANGE}, \texttt{TS\_ZSCORE}, \texttt{TS\_QUANTILE} & Rolling statistics computed along time axis per instrument \\
\midrule
Cross-Sectional & \texttt{RANK}, \texttt{ZSCORE}, \texttt{SCALE}, \texttt{MEAN}, \texttt{STD}, \texttt{MEDIAN}, \texttt{MAX}, \texttt{MIN}, \texttt{SKEW}, \texttt{KURT} & Statistics computed across stocks per datetime \\
\midrule
Mathematical & \texttt{ABS}, \texttt{SIGN}, \texttt{LOG}, \texttt{EXP}, \texttt{SQRT}, \texttt{POW}, \texttt{INV} & Element-wise mathematical functions \\
\midrule
Technical & \texttt{SMA}, \texttt{EMA}, \texttt{WMA}, \texttt{MACD}, \texttt{RSI}, \texttt{BB\_UPPER}, \texttt{BB\_LOWER}, \texttt{DECAYLINEAR}, \texttt{REGBETA}, \texttt{REGRESI} & Common technical indicators \\
\midrule
Logical & \texttt{GT}, \texttt{LT}, \texttt{GE}, \texttt{LE}, \texttt{AND}, \texttt{OR}, \texttt{WHERE} & Comparison and conditional operators \\
\midrule
Auxiliary & \texttt{COUNT}, \texttt{SUMIF}, \texttt{FILTER}, \texttt{PROD}, \texttt{HIGHDAY}, \texttt{LOWDAY} & Helper functions for complex expressions \\
\bottomrule
\end{tabular}
\end{table}

%%%%%%%%%%%%%%%%%%%%%%%%%%%%%%%%%%%%%%%%%%%%%%%%%%%%%%%%%%%%%%%%%%%%%%%%%%%%%%
\subsection{Trading Strategy Configuration}
\label{subsec:trading_strategy}
%%%%%%%%%%%%%%%%%%%%%%%%%%%%%%%%%%%%%%%%%%%%%%%%%%%%%%%%%%%%%%%%%%%%%%%%%%%%%%

We employ a TopkDropout strategy for portfolio construction, as detailed in Table~\ref{tab:trading_params}. On each trading day, stocks are ranked according to their predicted scores; the $n_{\text{drop}}$ lowest-scoring holdings are liquidated and replaced with the highest-ranked candidates to maintain a constant portfolio size with equal weighting.

\begin{table}[!htbp]
\centering
\caption{Parameters for the TopkDropout Trading Strategy}
\label{tab:trading_params}
\begin{tabular}{llp{5cm}}
\toprule
\textbf{Parameter} & \textbf{Value} & \textbf{Description} \\
\midrule
\multicolumn{3}{c}{\textit{Portfolio}} \\
\midrule
topk & 50 & Number of stocks held \\
n\_drop & 5 & Stocks dropped per rebalance \\
\midrule
\multicolumn{3}{c}{\textit{Transaction Costs}} \\
\midrule
Buying Fee & 0.05\% & Commission \\
Selling Fee & 0.15\% & Commission + stamp duty \\
\midrule
\multicolumn{3}{c}{\textit{Execution}} \\
\midrule
Deal Price & Open & Next-day opening price \\
Limit Threshold & 9.5\% & Price limit for halt \\
\midrule
\multicolumn{3}{c}{\textit{Benchmark}} \\
\midrule
China & SH000300/SH000905 & CSI 300/CSI500 \\
U.S. & SPX & S\&P 500 \\
\bottomrule
\end{tabular}
\end{table}
%%%%%%%%%%%%%%%%%%%%%%%%%%%%%%%%%%%%%%%%%%%%%%%%%%%%%%%%%%%%%%%%%%%%%%%
% NOTE: Add the following to your main document preamble:
%
% \usepackage{xcolor}
% \usepackage{tcolorbox}
% \tcbuselibrary{skins,breakable}
% \usepackage{tikz}
% \usepackage{fontawesome5}
% \usepackage{booktabs}
%
\definecolor{headerblue}{RGB}{41, 128, 185}
\definecolor{metricgreen}{RGB}{39, 174, 96}
\definecolor{alertred}{RGB}{192, 57, 43}
\definecolor{warningyellow}{RGB}{243, 156, 18}
\definecolor{lightblue}{RGB}{240, 245, 255}
\definecolor{darkblue}{RGB}{30, 60, 120}
\definecolor{mutationpurple}{RGB}{142, 68, 173}
\definecolor{crossoverblue}{RGB}{52, 152, 219}
%
%%%%%%%%%%%%%%%%%%%%%%%%%%%%%%%%%%%%%%%%%%%%%%%%%%%%%%%%%%%%%%%%%%%%%%%

\section{Case Study: Factor Evolution Trajectory}
\label{sec:appendix_case_study}

This appendix presents a detailed case study of factor evolution in QuantaAlpha. We trace the complete trajectory of a representative factor---\textit{Institutional\_Momentum\_Score\_20D}---through the crossover phase, demonstrating how the evolutionary framework synthesizes complementary market hypotheses from parent trajectories.

QuantaAlpha's evolution process operates in three phases: (1) \textbf{Original} phase where initial hypotheses are generated, (2) \textbf{Mutation} phase where existing trajectories are perturbed to explore orthogonal strategies, and (3) \textbf{Crossover} phase where high-performing parent trajectories are combined to synthesize offspring with potentially superior predictive power. The following factor card illustrates a Round 8 Crossover operation.

%%%%%%%%%%%%%%%%%%%%%%%%%%%%%%%%%%%%%%%%%%%%%%%%%%%%%%%%%%%%%%%%%%%%%%%
% Factor Identity Card
%%%%%%%%%%%%%%%%%%%%%%%%%%%%%%%%%%%%%%%%%%%%%%%%%%%%%%%%%%%%%%%%%%%%%%%
\subsection{Factor Identity}

The factor card below presents the basic information of the evolved factor, including its unique identifiers, evolution lineage, and mathematical formulation.

\begin{tcolorbox}[
  colback=lightblue,
  colframe=darkblue,
  fonttitle=\bfseries,
  title={\faTag\ Institutional\_Momentum\_Score\_20D},
  arc=2mm,
  boxrule=0.5pt,
  breakable
]

\begin{tabular}{@{}ll@{}}
\textbf{Factor ID:} & \texttt{c57cace576a95356} \\
\textbf{Trajectory ID:} & \texttt{df5a496878f4} \\
\textbf{Evolution Round:} & Round 8 \\
\textbf{Evolution Phase:} & \colorbox{crossoverblue!20}{\textcolor{crossoverblue}{\textbf{Crossover}}} \\
\textbf{Direction ID:} & 6 \\
\end{tabular}

\vspace{0.3cm}

\textbf{Factor Expression:}
\begin{tcolorbox}[colback=darkblue!5, colframe=darkblue, boxrule=0.3pt, arc=1mm]
\small\ttfamily
RANK(TS\_CORR(DELTA(close, 1)/close, DELTA(volume, 1)/volume, 20) * TS\_MEAN((close - open)/close, 5))
\end{tcolorbox}

\textbf{Mathematical Formulation:}
\[
\text{IMS}_{20D} = \text{RANK}\left(\rho_{20}\left(\frac{\Delta P}{P}, \frac{\Delta V}{V}\right) \times \overline{\left(\frac{C - O}{C}\right)}_{5}\right),
\]
\small
where $\rho_{20}(\cdot,\cdot)$ denotes the 20-day rolling correlation, $\Delta P/P$ is the daily return, $\Delta V/V$ is the volume change ratio, $\overline{(\cdot)}_5$ is the 5-day moving average, and $C$ and $O$ represent the closing and opening prices, respectively.

\vspace{0.2cm}

\textbf{Factor Interpretation:}\\
\small
This factor captures institutional-driven momentum by measuring two key signals: (1) the correlation between price returns and volume changes, which indicates coordinated institutional trading when positive; and (2) the average intraday return pattern, reflecting institutional activity that typically influences closing prices. The cross-sectional ranking ensures comparability across stocks.

\end{tcolorbox}

%%%%%%%%%%%%%%%%%%%%%%%%%%%%%%%%%%%%%%%%%%%%%%%%%%%%%%%%%%%%%%%%%%%%%%%
% Evolution Lineage
%%%%%%%%%%%%%%%%%%%%%%%%%%%%%%%%%%%%%%%%%%%%%%%%%%%%%%%%%%%%%%%%%%%%%%%
\subsection{Evolution Lineage}

The crossover operation combines insights from two parent trajectories with complementary market hypotheses. Parent 1 focuses on identifying \textit{fragile momentum} driven by retail speculation, while Parent 2 targets \textit{sustainable momentum} supported by institutional activity. The LLM synthesizes these complementary perspectives into a unified framework.

\begin{tcolorbox}[
  colback=lightblue,
  colframe=darkblue,
  fonttitle=\bfseries,
  title={\faDna\ Evolution Information},
  arc=2mm,
  boxrule=0.5pt,
  breakable
]

\textbf{\faCodeBranch\ Parent Trajectories:}

\vspace{0.3cm}

\begin{tcolorbox}[colback=lightblue, colframe=darkblue!50, boxrule=0.3pt, title={\small\textbf{Parent 1: 1e6d57e38e89}}, fonttitle=\bfseries\small]
\small
\begin{tabular}{@{}ll@{}}
\textbf{Round:} & Round 7 \\
\textbf{Phase:} & Mutation \\
\textbf{Rank IC:} & 0.0216 \\
\textbf{IC:} & 0.0059 \\
\textbf{IR:} & 1.297 \\
\end{tabular}

\vspace{0.2cm}
\textbf{Core Hypothesis:}\\
\small
When retail investors exhibit herd behavior and momentum chasing in stocks with high social media activity, but accompanied by declining institutional ownership and deteriorating fundamentals, the resulting price momentum is unsustainable and leads to mean reversion.
\end{tcolorbox}

\vspace{0.2cm}

\begin{tcolorbox}[colback=lightblue, colframe=darkblue!50, boxrule=0.3pt, title={\small\textbf{Parent 2: 47e0f0e55382}}, fonttitle=\bfseries\small]
\small
\begin{tabular}{@{}ll@{}}
\textbf{Round:} & Round 6 \\
\textbf{Phase:} & Crossover \\
\textbf{Rank IC:} & 0.0246 \\
\textbf{IC:} & 0.0069 \\
\textbf{IR:} & 1.347 \\
\end{tabular}

\vspace{0.2cm}
\textbf{Core Hypothesis:}\\
\scmall
A regime-adaptive structural momentum factor combining institutional ownership-driven medium-term price trends with short-term microstructure regime validation, where coordinated accumulation/distribution patterns amplify momentum when confirmed by microstructure alignment.
\end{tcolorbox}

\vspace{0.3cm}

\textbf{\faSitemap\ Evolution Path Diagram:}
\begin{center}
\resizebox{0.7\linewidth}{!}{%
\begin{tikzpicture}[
    node distance=1.5cm,
    roundbox/.style={rectangle, rounded corners, align=center, font=\scriptsize},
    parentbox/.style={roundbox, draw=darkblue!60, fill=lightblue, minimum width=2.5cm, minimum height=0.8cm},
    childbox/.style={roundbox, draw=darkblue, fill=darkblue!15, minimum width=3cm},
    arrow/.style={->, thick, >=stealth, color=darkblue}
]
    % Parent nodes
    \node[parentbox] (p1) at (-2.5, 1.8) {Parent 1\\Round 7 Mutation\\Rank IC: 0.0216};
    \node[parentbox] (p2) at (2.5, 1.8) {Parent 2\\Round 6 Crossover\\Rank IC: 0.0246};
    
    % Current node
    \node[childbox] (current) at (0, 0) {\textbf{Offspring Factor}\\Round 8 Crossover\\Rank IC: \textbf{0.0311}};
    
    % Arrows
    \draw[arrow] (p1) -- (current);
    \draw[arrow] (p2) -- (current);
    
    % Label
\end{tikzpicture}%
}
\end{center}

\end{tcolorbox}

%%%%%%%%%%%%%%%%%%%%%%%%%%%%%%%%%%%%%%%%%%%%%%%%%%%%%%%%%%%%%%%%%%%%%%%
% Synthesized Hypothesis
%%%%%%%%%%%%%%%%%%%%%%%%%%%%%%%%%%%%%%%%%%%%%%%%%%%%%%%%%%%%%%%%%%%%%%%
\subsection{Synthesized Hypothesis}

Through crossover, the LLM generates a new hypothesis that integrates the complementary insights from both parents, rather than simply averaging their factor expressions. This hypothesis-driven approach ensures that the offspring factor captures genuinely novel market dynamics.

\begin{tcolorbox}[
  colback=lightblue,
  colframe=darkblue,
  fonttitle=\bfseries,
  title={\faLightbulb\ Hypothesis},
  arc=2mm,
  boxrule=0.5pt,
  breakable
]

\textbf{Core Hypothesis:}\\
A regime-aware dual-source momentum factor that combines institutional-driven structural momentum (validated by healthy microstructure) and retail-driven speculative momentum (characterized by high attention and deteriorating fundamentals), dynamically weighted by market volatility: amplifying institutional signals in stable regimes and retail reversal signals in turbulent regimes, will generate superior predictive returns.

\vspace{0.3cm}

\begin{tabular}{@{}p{0.28\textwidth}p{0.67\textwidth}@{}}
\toprule
\textbf{Component} & \textbf{Description} \\
\midrule
\textbf{Observation} & Parent strategies separately targeting institutional trends and retail herding show moderate predictive power (Rank IC $\sim$0.02--0.025), suggesting combined signals could capture complementary market dynamics. \\
\midrule
\textbf{Justification} & Sustainable price trends require institutional sponsorship and orderly trading, while retail-driven bubbles lack fundamental support and reverse under stress; a hybrid model exploiting both can enhance robustness across market regimes. \\
\midrule
\textbf{Domain Knowledge} & Institutional accumulation with strong price-volume correlation and low volatility indicates sustainable momentum; retail herding with declining institutional ownership and high volatility signals fragile momentum prone to reversal. \\
\bottomrule
\end{tabular}

\end{tcolorbox}

%%%%%%%%%%%%%%%%%%%%%%%%%%%%%%%%%%%%%%%%%%%%%%%%%%%%%%%%%%%%%%%%%%%%%%%
% Backtest Results
%%%%%%%%%%%%%%%%%%%%%%%%%%%%%%%%%%%%%%%%%%%%%%%%%%%%%%%%%%%%%%%%%%%%%%%
\subsection{Backtest Performance}

After factor construction, QuantaAlpha automatically backtests the generated factors using the Qlib framework. The results below compare the offspring factor against both parent trajectories and the baseline, demonstrating the effectiveness of the crossover operation.

\begin{tcolorbox}[
  colback=lightblue,
  colframe=darkblue,
  fonttitle=\bfseries,
  title={\faChartLine\ Backtest Metrics},
  arc=2mm,
  boxrule=0.5pt,
  breakable
]

\begin{center}
\begin{tabular}{@{}lccc@{}}
\toprule
\textbf{Metric} & \textbf{Offspring Factor} & \textbf{Baseline} \\
\midrule
\textbf{IC} & 0.0126 & 0.0058 \\
\textbf{Rank IC} & \textbf{0.0311} & 0.0220 \\
\midrule
\textbf{ARR (Excess)} & 7.80\% & 5.20\% \\
\textbf{IR} & 0.963 & 0.973 \\
\textbf{MDD (Excess)} & $-$11.37\% & $-$7.30\% \\
\bottomrule
\end{tabular}
\end{center}

\vspace{0.3cm}

\textbf{Detailed Statistics:}
\begin{center}
\small
\begin{tabular}{@{}ll|ll@{}}
\toprule
\textbf{Metric} & \textbf{Value} & \textbf{Metric} & \textbf{Value} \\
\midrule
\textbf{Daily Excess Return (w/o cost)} & 0.0328\% & \textbf{Daily Excess Return (w/ cost)} & 0.0128\% \\
\textbf{Excess Return Std} & 0.52\% & \textbf{Turnover (FFR)} & 100\% \\
\textbf{L2 Train Loss} & 0.9936 & \textbf{L2 Valid Loss} & 0.9962 \\
\bottomrule
\end{tabular}
\end{center}

\end{tcolorbox}

%%%%%%%%%%%%%%%%%%%%%%%%%%%%%%%%%%%%%%%%%%%%%%%%%%%%%%%%%%%%%%%%%%%%%%%
% LLM Feedback
%%%%%%%%%%%%%%%%%%%%%%%%%%%%%%%%%%%%%%%%%%%%%%%%%%%%%%%%%%%%%%%%%%%%%%%
\subsection{LLM Feedback and Iteration Guidance}

After evaluating backtest results, the LLM provides structured feedback that guides subsequent evolution rounds. This feedback loop enables continuous improvement by learning from both successes and failures.

\begin{tcolorbox}[
  colback=lightblue,
  colframe=darkblue,
  fonttitle=\bfseries,
  title={\faComments\ Feedback \& Evaluation},
  arc=2mm,
  boxrule=0.5pt,
  breakable
]

\textbf{\faSearch\ Observations:}\\
The crossover operation demonstrates a trade-off between enhanced predictive accuracy and increased risk exposure compared to the baseline:
\begin{itemize}
    \item \textcolor{metricgreen}{\faCheckCircle} Significant improvement in annualized excess return and predictive metrics (IC and Rank IC), validating the effectiveness of synthesizing dual-source momentum signals.
    \item \textcolor{alertred}{\faTimesCircle} Increased maximum drawdown and a marginal decline in the Information Ratio, suggesting that the offspring factor introduces higher volatility during certain market regimes.
\end{itemize}

\vspace{0.2cm}

\textbf{\faBalanceScale\ Hypothesis Evaluation:}\\[0.15cm]
\small
Results partially support the hypothesis. Improved annualized return and IC suggest that combining institutional and retail momentum signals has merit. However, deterioration in risk metrics indicates that without proper regime-adaptive weighting, the combined signals may amplify risks during turbulent periods. The full hypothesis requires all three components (institutional momentum, retail herding reversal, volatility-adaptive weighting) to work effectively.

\vspace{0.2cm}

\textbf{\faFlag\ Decision:} \colorbox{alertred!20}{\textcolor{alertred}{\textbf{REJECTED}}} for direct deployment.

\vspace{0.2cm}

\textbf{\faExclamationTriangle\ Recommendations for Next Iteration:}
\begin{enumerate}
    \setlength{\itemsep}{0pt}
    \setlength{\parsep}{0pt}
    \setlength{\topsep}{0pt}
    \setlength{\partopsep}{0pt}
    \small
    \item Use 20-day price-volume correlation as institutional momentum proxy;
    \item Use 5-day average intraday returns as retail attention proxy;
    \item Add volatility regime indicator (recent/historical volatility ratio) for dynamic weighting.
\end{enumerate}

\small This feedback will inform the next mutation round, guiding the LLM to simplify the factor expression while preserving the core dual-source concept.

\end{tcolorbox}

%%%%%%%%%%%%%%%%%%%%%%%%%%%%%%%%%%%%%%%%%%%%%%%%%%%%%%%%%%%%%%%%%%%%%%%%%%%%%%
%%%%%%%%%%%%%%%%%%%%%%%%%%%%%%%%%%%%%%%%%%%%%%%%%%%%%%%%%%%%%%%%%%%%%%%%%%%%%%
\section{Appendix: Alpha Decay}
\label{app:factor_analysis}

This appendix provides a factor-level interpretation of why QuantaAlpha remains effective during the 2023 regime shift, while baseline methods fail.
We analyze both market microstructure changes and the semantic categories of dominant factors.

\subsection{The 2023 Regime Shift in the A-share Market}

The divergence in performance originates from a mismatch between the training distribution (2016--2020) and the 2023 market environment.

\begin{itemize}
    \setlength{\itemsep}{0pt}
    \setlength{\parskip}{0pt}
    \item \textbf{Training Regime (Large-Cap Dominance).}
    The CSI~300 during 2016--2020 is characterized by large-cap leadership, high institutional participation, and relatively low intraday noise.
    Trend-following and classical mean-reversion signals are stable under this regime.
    
    \item \textbf{Test Regime (Small-Cap Rotation in 2023).}
    In 2023, market liquidity becomes fragmented and rotates toward small-cap and thematic stocks.
    This regime exhibits:
    (i) noisy intraday price movements,
    (ii) frequent and information-rich overnight gaps driven by call auction activity,
    and (iii) gradual liquidity accumulation preceding price breakouts.
\end{itemize}

Such conditions invalidate the core assumptions embedded in many baseline factor designs.

\subsection{Structural Factor Semantics and Regime Robustness}

The 2023 performance contrast highlights that robust alpha generation depends on identifying \emph{structural information channels} that remain stable across liquidity regimes:

\label{subsec:qa_structural_channels}

\begin{itemize}
    \setlength{\itemsep}{2pt}
    \setlength{\parskip}{0pt}
    \item \textbf{Overnight information channel (Gap Factors)}: Factors such as \texttt{GapZ10\_Overnight\_vs\_TR} capture the risk premium associated with non-trading hours. In 2023, as intraday predictability weakened, overnight price formation became a dominant information channel, rendering gap signals highly effective at quantifying off-hours information processing.
    \item \textbf{Volatility structure channel (Range Deviation)}: Signals like \texttt{Mean-Reverting Range Deviation} focus on abnormal intraday range expansion relative to volatility clusters. Instead of predicting simple direction, these primitives identify transient dislocations in price variability, which remain reliable even under noisy trading conditions.
    \item \textbf{Trend-quality channel (clean trend vs.\ pseudo-trend)}: QuantaAlpha evolves factors that explicitly penalize residual volatility and unstable volume rather than relying on raw momentum. This filtering separates orderly trends with strong participation from the noisy pseudo-trends driven by retail liquidity in small-cap stocks.
    \item \textbf{Baseline reversal/exhaustion channel}: Traditional baseline methods rely on \emph{volume-price exhaustion} or \emph{climax-reversal} logic. While effective in institutionally dominated markets, this channel often breaks down in retail-driven regimes (like 2023), where overbought conditions persist due to thematic narratives rather than mean-reverting immediately.
\end{itemize}

The discovery of these primitives suggests two high-level design principles for temporal robustness:
\textbf{Prioritize semantic stability over style.} Baseline failures highlight the fragility of factors tied to specific market styles (e.g., rapid reversal). In contrast, economically interpretable primitives—such as overnight risk and volatility structure—generalize better across OOD data.
\textbf{Avoid homogeneity in signal logic.} Relying solely on price-volume exhaustion creates semantic homogeneity; incorporating orthogonal channels like trend quality and overnight gaps ensures sustained alpha generation when dominant market dynamics shift.

% Appendix (Factor-level diagnosis): QA vs AA in 2023
% This file is intended to be included via \input{}; do NOT add documentclass/preamble here.

\section{Factor-Level Diagnosis in 2023 (CSI300): QuantaAlpha vs.\ AlphaAgent}
\label{sec:appendix_factor_diagnosis_quantaalpha_vs_alphaagent_2023}

This appendix provides a factor-level diagnosis for the year 2023 on the CSI300 universe. The goal is to connect the observed strategy-level gap to \emph{which information channels were predictive} under the 2022--2023 regime transition, rather than to exhaustively list factor implementations.

\label{subsec:appendix_market_context_2023}

Market microstructure and style exposures in A-shares changed materially during 2022--2023. In particular, dispersion and rotation across styles increased, and short-horizon reversal effects became more pronounced in several sub-periods. Meanwhile, the growth of programmatic trading (e.g., DMA-style execution and increased private-fund participation) amplified the importance of \emph{auction/overnight information}, \emph{liquidity re-rating}, and \emph{trend quality filtering} (distinguishing ``clean'' trends from noise-driven moves). Under such non-stationarity, factors that rely on a single, narrow hypothesis are more likely to become unstable.

\label{subsec:appendix_summary_stats_2023}

Table~\ref{tab:qa_aa_2023_summary} reports the absolute summary statistics of annual Rank IC and IC in 2023. We use the \textbf{coverage ratio} to describe the fraction of factors with valid annual metrics (instead of enumerating factor counts).

\makeatletter
\setlength{\@fptop}{10pt}
\makeatother

\begin{table}[!htbp]
\centering
\caption{Summary statistics of annual factor predictability on the CSI 300 index in 2023.}
\label{tab:qa_aa_2023_summary}
\small
\setlength{\tabcolsep}{5pt}
\begin{tabular}{lcc}
\toprule
\textbf{Metric} & \textbf{QA} & \textbf{AA} \\
\midrule
Coverage ratio (valid metrics) & 0.98 & 0.80 \\
Share with Rank IC $>0$ & 0.626 & 0.594 \\
Mean Rank IC & 0.0057 & 0.0012 \\
Max Rank IC & 0.0793 & 0.0323 \\
Min Rank IC & -0.0720 & -0.0279 \\
Share with Rank IC $>0.03$ & 0.102 & 0.0156 \\
Share with Rank IC $>0.05$ & 0.0272 & 0.0000 \\
\midrule
Mean IC & 0.0044 & 0.0015 \\
\bottomrule
\end{tabular}
\end{table}

To keep the appendix concise, we highlight representative factors at the extremes of 2023 Rank IC for each method (Table~\ref{tab:qa_aa_2023_representative}). We do not repeat full formulas; each entry is summarized by its economic meaning.

\label{subsec:appendix_dominant_signals_2023}

\begin{table}[!htbp]
\centering
\caption{Representative factors and their performance metrics on the CSI 300 index in 2023.}
\label{tab:qa_aa_2023_representative}
\scriptsize
\setlength{\tabcolsep}{4pt}
\begin{tabular}{p{7cm}ccp{6.3cm}}
\toprule
\textbf{Factor (representative)} & \textbf{Rank IC} & \textbf{IC} & \textbf{Interpretation (short)} \\
\midrule
\multicolumn{4}{c}{\textit{QA --- strong performers (overnight/auction, trend-quality, and liquidity re-rating)}} \\
\midrule
\texttt{GapZ10\_Overnight\_vs\_TR} & 0.0793 & 0.0335 & Normalized overnight gap magnitude relative to recent true range; captures auction-driven shocks and subsequent adjustment. \\
\texttt{Gap\_IntradayAcceptanceScore\_20D} & 0.0744 & 0.0330 & ``Acceptance vs.\ rejection'' of an overnight gap using intraday direction, scaled by recent volatility. \\
\texttt{Gap\_IntradayAcceptance\_VolWeighted\_20D} & 0.0606 & 0.0314 & Gap acceptance score weighted by abnormal volume; emphasizes information-rich openings with high participation. \\
\texttt{CleanTrend\_Continuation\_Score\_RS10\_KLEN10\_WVMA5} & 0.0590 & 0.0267 & Trend continuation conditioned on high trend quality (low residual noise) and muted intraday/volume pressure. \\
\texttt{OrderlyTrend\_x\_Absorption\_10D\_5D\_20D} & 0.0465 & 0.0271 & ``Orderly'' short-horizon trend cross-validated by liquidity absorption (high dollar volume with low price impact). \\
\midrule
\multicolumn{4}{c}{\textit{QA --- weak performers (overly rigid gates or noisy-path proxies under rotation)}} \\
\midrule
\texttt{KineticLength\_AbsRetSum\_Z\_10D} & -0.0720 & -0.0246 & Path-length proxy (choppiness): can behave like a noise detector, but may invert under fast style rotation. \\
\texttt{Drawdown\_Gated\_NegCorr\_60D\_20D\_thr20pct} & -0.0282 & -0.0095 & Hard regime gate based on deep drawdown; brittle when drawdowns cluster and cross-sectional regimes shift quickly. \\
\texttt{HighClose\_Shock\_With\_VolSync\_60\_20} & -0.0274 & -0.0090 & ``Shock-day'' breakout quality (close-in-range, range shock, return--volume sync); sensitive to regime-dependent follow-through. \\
\midrule
\multicolumn{4}{c}{\textit{AA --- strong performers (exhaustion/climax-style reversals)}} \\
\midrule
\texttt{Exhaustion\_Intensity\_Index\_10D} & 0.0323 & 0.0159 & Price displacement over 60D interacted with volume intensity ratio; targets potential exhaustion and reversal. \\
\texttt{Climax\_Exhaustion\_Intensity} & 0.0242 & 0.0160 & Variant using short-horizon volume climax vs.\ long-horizon baseline; aims to identify capitulation-like turns. \\
\texttt{Exhaustion\_Volume\_Instability\_Index} & 0.0121 & 0.0117 & Trend deviation combined with volume instability; highlights fragile price levels supported by unstable liquidity. \\
\midrule
\multicolumn{4}{c}{\textit{AA --- weak performers (bottom-fishing under non-stationary liquidity)}} \\
\midrule
\texttt{Relative\_Volume\_Calm\_Reversal} & -0.0279 & -0.0188 & ``Quiet-volume'' regimes multiplied by momentum divergence; may fail when liquidity conditions change abruptly. \\
\texttt{Volume\_Stability\_Momentum\_Divergence\_40D} & -0.0247 & -0.0155 & Robust volume-stability proxy (MAD) times momentum spread; sensitive to turnover regime changes. \\
\texttt{LVR\_Bottom\_Fishing\_20D} & -0.0190 & -0.0144 & ``Bottom-fishing'' reversal with intraday rejection and volume surge; vulnerable when reversals are short-lived and crowded. \\
\bottomrule
\end{tabular}
\end{table}

The 2022--2023 regime transition emphasizes that predictive signals should be interpreted as \emph{information channels} with different stability across regimes:

\label{subsec:appendix_interpretation_channels}

\begin{itemize}
    \setlength{\itemsep}{2pt}
    \setlength{\parskip}{0pt}
    \item \textbf{Overnight/auction channel (Gap)}: Overnight gaps aggregate information released off-hours and the supply--demand balance during the opening auction. In 2023, representative gap-based signals (Table~\ref{tab:qa_aa_2023_representative}) indicate that this channel can be informative even when intraday price action is noisy.
    \item \textbf{Trend-quality channel (clean trend vs.\ noise)}: Under higher dispersion and rapid rotation, ``plain momentum'' is often unstable; conditioning on trend quality (e.g., low residual volatility, muted intraday pressure) helps separate persistent flows from noise-driven moves.
    \item \textbf{Liquidity re-rating channel (dollar-volume up, price-impact down)}: When capital reallocates quickly across themes, changes in liquidity and price impact can lead price moves. Signals combining volume expansion with improving liquidity can therefore be predictive in trend-following and rotation regimes.
    \item \textbf{Pure reversal/exhaustion channel}: Reversal-style constructs can work episodically, but their stability depends on how quickly liquidity replenishes after shocks and whether the trades become crowded. Under non-stationary liquidity, rigid ``bottom-fishing'' patterns may be less reliable.
\end{itemize}

The 2023 diagnosis suggests two high-level design principles (without duplicating the main-text methodology):
\textbf{Prefer multi-channel confirmation.} Combining orthogonal channels (overnight/auction, trend quality, liquidity) can reduce regime sensitivity relative to single-hypothesis factors.
\textbf{Avoid brittle hard gates.} Hard thresholds (e.g., deep drawdown triggers) can be fragile when cross-sectional regimes shift quickly; smoother conditioning is often more robust.


\section{Iteration Study: Convergence of Factor-Pool Performance (CSI300)}
\label{sec:appendix_iteration_convergence}

This appendix studies how the strategy-level performance of the \textbf{maintained high-quality factor pool} evolves as QuantaAlpha runs for more iterations on CSI300. The first five iterations are reported in Section~5.5 of the main text; here we focus on \textbf{Iter 5--15} to examine whether performance converges or starts to deteriorate.

\label{subsec:appendix_iteration_setup}

We run QuantaAlpha for a total of 15 iterations on CSI300, using \textbf{Deepseek-V3.2} as the backbone LLM for all agents. To improve mining throughput, we set the factor-complexity constraints to \textbf{symbol length $\leq 200$} and \textbf{base features $\leq 4$}. Each iteration consists of two phases: a \textit{Mutation} phase followed by a \textit{Crossover} phase.

At the end of each iteration, we maintain a global \textbf{high-quality factor pool} using all factors generated up to that iteration. The maintenance rule is greedy and RankIC-driven: we sort all candidate factors by RankIC (descending) and add them into the pool in order, subject to a redundancy constraint. A factor is admitted only if its \textbf{absolute correlation} with every factor already in the pool is \textbf{below 0.7}. The pool size is capped at \textbf{50\% of the total number of mined factors} up to the current iteration. 

\begin{figure}[t]
\centering
\includegraphics[width=1\columnwidth]{images/iteration.png}
\caption{Factor Pool Performance by Iteration.}
\label{fig:annual_ic_rankic}
\end{figure}


\label{subsec:appendix_iteration_arr_mdd}

In the two-column paper draft, we keep this appendix lightweight and report the key iteration-wise conclusions using absolute values. (The corresponding figure and compact tables can be added later; they are intentionally omitted here to avoid path and float issues when this file is included from different build roots.)

\label{subsec:appendix_iteration_observations}

Across Iter 5--15, ARR increases from \textbf{29.63\%} (Iter 5) to \textbf{35.97\%} (Iter 15), with a clear step-up at Iter 10 (\textbf{34.37\%}) and a peak at \textbf{36.02\%} (Iter 11). Meanwhile, the drawdown magnitude improves until Iter 12, where $|\mathrm{MDD}|$ reaches its minimum (\textbf{7.58\%}), and then increases afterwards (e.g., \textbf{9.36\%} at Iter 15). Overall, the curve suggests that performance \textbf{approaches a plateau around Iter 11--13}: further iterations yield limited incremental ARR while the drawdown profile can become less favorable. In practice, selecting \textbf{Iter 11--12} offers a balanced trade-off between return and drawdown under this setup.







\end{document}


